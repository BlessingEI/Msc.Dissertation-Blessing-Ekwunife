\chapter*{\Large \center Abstract}

This research investigates the application of machine learning techniques for detecting malware in Internet of Things (IoT) network traffic. As IoT devices proliferate across homes and critical infrastructure, they present attractive targets for cyber attacks due to their inherent security limitations. This study leverages the CTU-IoT-Malware dataset, which contains approximately one million labelled network connections, to develop and evaluate models capable of distinguishing between benign and malicious traffic patterns.

The research implements a comprehensive methodology that involves extensive data pre-processing, feature engineering, and the development of multiple machine learning models. Exploratory data analysis revealed distinct patterns in protocol usage, connection states, and temporal behaviour between benign and malicious traffic. TCP dominated malicious traffic (72.4\% vs 40.9\% in benign), with the state of connection S0 (connection attempt without reply) representing 94. 7\% of malicious connections, strongly indicative of scanning behaviour.

Four classification approaches were implemented and compared: Random Forest, Support Vector Machine (SVM), XGBoost, and a custom neural network architecture. The models were rigorously evaluated using metrics including accuracy, precision, recall, F1 score, and area under the ROC curve. The optimised Random Forest model achieved the highest overall performance with 99.8\% accuracy and 0.998 F1 score, slightly outperforming the neural network (99.6\% accuracy). Feature importance analysis revealed that connection state, protocol type, and metrics derived such as bytes per packet ratios were particularly influential in classification decisions. The research contributes valuable information for the development of lightweight and effective IoT security monitoring solutions by identifying the most discriminative features of network traffic and quantifying the effectiveness of various machine learning approaches for malware detection in resource-constrained environments.

\vspace{0.5cm}
\textbf{Keywords:} IoT Security, Network Traffic Analysis, Machine Learning, Malware Detection, Random Forest, Deep Learning
