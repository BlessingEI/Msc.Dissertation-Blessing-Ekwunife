\chapter{Critical Appraisal}

This chapter offers a personal reflection on the research project focused on using machine learning for network malware detection in IoT environments. Drawing on my experience in cybersecurity, we discuss the lessons learnt, the benefits of the project, its key insights, and the aspects I would approach differently in future endeavours.

Working with the CTU-IoT-Malware dataset was an eye-opening experience. I encountered firsthand the challenges inherent in managing real-world data, particularly the issues of imbalanced samples and missing values. Although the dataset impressed with its scale and diversity, its strong bias toward scanning activities and outdated information limited its ability to represent the evolving tactics of modern malware. This project underscored the need for regularly updated datasets and a broader range of attack types to more accurately simulate the current threat landscape.

The feature engineering process provided valuable information. I appreciated how exploratory analysis could pinpoint discriminative features, such as connection states and protocol usage. Nevertheless, I also recognised the risk of overreliance on a limited set of features, which may fail to capture the complexities of sophisticated attacks. In future projects, I plan to employ more advanced packet inspection techniques and incorporate additional metadata to develop more resilient detection systems.

Experimenting with various machine learning models, from traditional algorithms to neural networks, deepened my understanding of the impact of hyperparameter tuning and the pitfalls of overoptimisation. Although achieving high accuracy was encouraging, it raised important questions about possible data leakage and overfitting. I learnt that balancing performance metrics is crucial, especially in cybersecurity, where false negatives can have severe consequences. Therefore, future efforts will focus on minimising undetected threats without compromising overall model performance.

The project also highlighted the importance of deploying machine learning in real-world security applications, emphasising interpretability and real-time performance. It broadened my appreciation not only for high-accuracy models but also for the need for transparent, explainable systems, for instance, through techniques like SHAP values or partial dependence plots. These practical insights are vital for designing systems capable of effective operation in resource-constrained IoT environments.

On a theoretical level, comparing different modelling approaches enriched my understanding of how various machine learning paradigms address cybersecurity data. Furthermore, grappling with ethical considerations, such as privacy concerns and the risk of dual use, underscored the responsibility that accompany the deployment of advanced technologies. Now I appreciate the importance of aligning security innovations with ethical guidelines and the need for clear model interpretability.

As a cybersecurity expert, I found that mastering certain data science concepts was challenging. However, my background in computer science helped bridge the gap between theory and practice, enabling me to overcome these obstacles. In addition, effective time management and clear communication with my supervisor were essential in navigating the complex challenges encountered throughout this project.
